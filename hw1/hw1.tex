%%%%%%%%%%%%%%%%%%%%%%%%%%%%%%%%%%%%
% Alexander Powell
% CSCI 654 - Advanced Computer Architecture
% Homework #1
% Due: 09.16.2016
%%%%%%%%%%%%%%%%%%%%%%%%%%%%%%%%%%%%

\documentclass[10pt]{article} %
\usepackage{fullpage}
\usepackage{graphicx}
\usepackage{graphics}
\usepackage{psfrag}
\usepackage{amsmath,amssymb}
\usepackage{enumerate}
\usepackage[makeroom]{cancel}

\usepackage{mathtools}
\DeclarePairedDelimiter{\ceil}{\lceil}{\rceil}
\DeclarePairedDelimiter{\floor}{\lfloor}{\rfloor}

\setlength{\textwidth}{6.5in}
\setlength{\textheight}{9in}

\newcommand{\cP}{\mathcal{P}}
\newcommand{\N}{\mathbb{N}}
\newcommand{\Z}{\mathbb{Z}}
\newcommand{\R}{\mathbb{R}}
\newcommand{\Q}{\mathbb{Q}}
\newcommand{\points}[1]{{\it (#1 Points)}}
\newcommand{\tpoints}[1]{{\bf #1 Total points.}}

\title{CSCI 654 -- Advanced Computer Architecture \\
Homework 1 \\
{\large{\bf Due: September 16, 2016}}}
\date{}
\author{Alexander Powell}


\begin{document}
\maketitle
\begin{enumerate}

\item %1

\begin{enumerate}[(a)]
\item

From Amdahl's law we have that $\dfrac{1}{(1-p) + p/s}$.  By plugging in the values from the problem we see $\dfrac{1}{0.3 + 0.7/2} = \dfrac{1}{0.65}$.  Since $1 - 0.65 = 0.35$ you can decrease the frequency by $35\%$ and still get the same performance.  

\item

From the power equation, we know that 
$$ \dfrac{\text{new power}}{\text{old power}} = 2 \times \dfrac{(\text{Voltage} \times 0.65)^2 \times \text{Frequency} \times 0.65}{\text{Voltage}^2 \times \text{Frequency}} $$

By cancelling out the common terms in the fraction, we get $0.54925$.  

\item

To find the parallelization, we set up the following equation with the result of part b.  
$$ 0.54925 = 2 \times \dfrac{(\text{Voltage} \times 0.3)^2 \times{Frequency} \times X}{\text{Voltage}^2 \times \text{Frequency}} $$
$$ X = \dfrac{0.54925}{2 \times 0.3^2} $$
Then we have that $3.05138 = \dfrac{1}{(1-p) + p/2}$.  Solving this gives us $p \approx 0.34456$ so about $34\%$ parallelization gives us a voltage at the voltage floor.  

\item

Again, from the equation in part b, we see that
$$ \dfrac{\text{new power}}{\text{old power}} = 2 \times \dfrac{(\text{Voltage} \times 0.3)^2 \times \text{Frequency} \times 0.65}{\text{Voltage}^2 \times \text{Frequency}} $$

After cancelling, this gives us $0.117$.  

\end{enumerate}

\item %2

\begin{enumerate}

\item

Since the FIT for a single computer is given as $\dfrac{300 \text{ failures}}{\text{1 billion hours}}$, then the MTTF can be calculated as $\dfrac{10000000\cancel{00}}{3\cancel{00}} = 3,333,333.3333 \text{ hours}$.  $2/5$ of the $20000$ machines is $8000$, so the MTTF for the system can be calculated as $\dfrac{10000000}{3} \times 8000 = 26,666,666,666 \text{ hours}$.  

\item

Since the FIT is given as $\dfrac{300 \text{ failures}}{1 \text{ billion hours}}$, and 1 billion hours is equal to $41,666,666$ days, then we find that $7.2 \times 10^{-6}$ failures occur each day.  Let's denote the cost to repair a computer as $C$, then the amount of money lost, per day, to computers failing is $\$7.2 \times 10^{-6}$.  

\end{enumerate}

\item %3

\begin{enumerate}

\item $\dfrac{1}{0.2 + 0.8/N}$

\item $\dfrac{1}{0.2 + 0.8/8 + 8 \times 0.005} \approx 2.94118$

\item To go from 1 to 8 processors, the number of processors doubles three times (once from 1 to 2, once from 2 to 4, and finally from 4 to 8).  We can modify the previous answer to get: $$\dfrac{1}{0.2 + 0.8/8 + 3 \times 0.005} \approx 2.94118$$

\item To generalize the above question to find the number of times $N$ doubles we can use $\log_2$.  Thus, like above, we have:
$$ \dfrac{1}{0.2 + 0.8/N + \floor*{\log_2(N)} \times 0.005} $$
Note: the floor function, $\floor*{x}$ is used here to denote that we're rounding down the result of the logarithm.  

\end{enumerate}

\end{enumerate}
\end{document}















